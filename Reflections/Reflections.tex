% The format (45) is selected to facilitate reading on small devices
\documentclass[a5paper,10pt,oneside]{article}

\usepackage[swedish]{babel}
\usepackage[T1]{fontenc}

% Change latin1 here to the encoding you use
\usepackage[utf8]{inputenc}


% The packages listed below are optional and can be removed if you don't use them
\usepackage{graphicx}
\usepackage{cite}
\usepackage{url}
\usepackage{ifthen}
\usepackage{listings}	



\usepackage{ifpdf}
\ifpdf
	\usepackage[hidelinks]{hyperref}
\else
	\usepackage{url}
\fi



\title{Assignment 2: Reflections}


\author{Oscar Ferm \and Mattias Forsman \and Frans Nordén}




\begin{document}

\maketitle
\pagebreak

\section{Reflektioner gällande macron}

\subsection{Safe macro}
Detta macro skulle inte fungera att definera som en funktion eftersom argumenten som skickas in inte kan utvärderas direkt utan 
komplikationer. Detta är nämligen en av skillnaderna mellan ett macro och en funktion; att argumenten till en funktion utvärderas när
de skickas in till funktionen emedan de för ett macro ej gör det. I detta fall så skulle man ifall argumenten utvärderades direkt 
få problem med att exceptions uppstår som aldrig fångas. Om man skickar in en fil och får FileNotFoundException från File men också 
har öppnat en FileReader så skulle man med en funktion ej kunna stänga FileReadern, vilket man med detta macro alltid gör.

\subsection{SQL-like macro}
Detta macro anser vi att man bör kunna implementera som en funktion eftersom ingenting tar skada av att argumenten utvärderas direkt när funktionen
anropas. Det som skulle vara fördelen med att definera ett macro för denna typ av uppgift är att man då kan definera en typ av "sub-språk" inuti det programeringsspråk man använder vilket kan öka readability och writeability. 

% Vad som kan påverka??
% 



\end{document}